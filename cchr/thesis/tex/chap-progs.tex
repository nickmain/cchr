\chapter{Gebruikte programma's} \label{chap:progs}

Hier wordt een overzicht gegeven van alle software die gebruikt werd bij het maken van deze thesis: \begin{itemize}
\item De benchmarks zijn uitgevoerd op een Gentoo Linux systeem, gebruik makende van een Linux 2.6.19 kernel.
\item De GCC (GNU Compiler Collection) versie 4.1.2 werd gebruikt om de CCHR compiler en de C en CCHR voorbeeldprogramma's te compileren.
\item Java 1.5.0.11 compiler en runtime en het K.U.Leuven JCHR systeem v1.5.1 werden gebruikt voor de voor de JCHR voorbeelden.
\item SWI-Prolog v5.6.17 werd gebruikt voor de Prolog-CHR voorbeelden.
\item De C, CCHR en Prolog voorbeelden gebruikten GMP (GNU Multiprecision Library) versie 4.2.1.
\item GNU Flex 2.5.33 werd gebruikt voor het genereren van de lexer.
\item GNU Bison 2.3 werd gebruikt voor het genereren van de parser.
\item Gnuplot 4.2rc4 werd gebruikt voor het maken van de grafieken.
\item Het schrijven van de code gebeurde met GNU Midnight Commander 4.6.1, Eclipse 3.1 en Kate 2.5.6.
\item Het typesetten van deze thesis gebeurde met de latex-distributie texlive 3.141592-1.30.5-2.2.
\end{itemize}
